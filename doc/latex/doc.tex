\documentclass[11pt,a4paper,DIV=10,]{scrartcl}
\usepackage[utf8]{inputenc}
\usepackage[ngerman]{babel}
\usepackage{amsmath}
\usepackage{amsfonts}
\usepackage{amssymb}
\usepackage{amsthm}
\usepackage{fancybox}
\usepackage{multicol}
\usepackage{graphicx}
\usepackage{float}
\usepackage{listings}
\usepackage{color}
\usepackage{colortbl}

% Define user colors using the RGB model
\definecolor{dunkelgrau}{rgb}{0.8,0.8,0.8}
\definecolor{hellgrau}{rgb}{0.95,0.95,0.95}
\definecolor{middlegray}{rgb}{0.5,0.5,0.5}
\definecolor{lightgray}{rgb}{0.8,0.8,0.8}
\definecolor{orange}{rgb}{0.8,0.3,0.3}
\definecolor{yac}{rgb}{0.6,0.6,0.1}

% Zitation und Literaturverzeichnis
\usepackage[normal,font={small,color=black}, labelfont=bf,figurename=Abb.]{caption}
\usepackage{cite}
\usepackage{url}
\bibliographystyle{unsrtnat}
\usepackage[numbers]{natbib}
%\usepackage[T1]{fontenc}

\begin{document}
% Formatierung für das Listing
\lstset{
   basicstyle=\scriptsize\ttfamily,
   keywordstyle=\bfseries\ttfamily\color{orange},
   stringstyle=\color{green}\ttfamily,
   commentstyle=\color{middlegray}\ttfamily,
   emph={square}, 
   emphstyle=\color{blue}\texttt,
   emph={[2]root,base},
   emphstyle={[2]\color{yac}\texttt},
   showstringspaces=false,
   flexiblecolumns=false,
   tabsize=2,
   numbers=left,
   numberstyle=\tiny,
   numberblanklines=true,
   stepnumber=1,
   numbersep=10pt,
   xleftmargin=15pt
}

\subsection*{DBS SoSe 2013, Di. 8-10}
\section*{Projektdokumentation: Iteratation No. 2}
\textbf{Christoph van Heteren-Frese (Matr.-Nr.: 4465677), \\ Sven Wildermann (Matr.-Nr.: 4567553) }
\hrule
\section*{Erzeugen der Relationen/Tabellen}

\subsection*{Beispiel: geodb\_coordinates}
\begin{lstlisting}[language=sql]
CREATE TABLE geodb_coordinates (
	loc_id integer not null,
	coord_type integer not null,
	lat numeric(10,8),
	lon numeric(11,8),
	coord_subtype integer,
	valid_since integer,
	data_type_since integer,
	valid_until date not null,
	date_type_until integer not null
);
\end{lstlisting}
\subsection*{Beispiel: zip\_coordinates}
\begin{lstlisting}[language=sql]
CREATE TABLE zip_coordinates (
    zc_id serial NOT NULL PRIMARY KEY,
    zc_loc_id INT NOT NULL ,                                            
    zc_zip VARCHAR( 10 ) NOT NULL ,
    zc_location_name VARCHAR( 255 ) NOT NULL ,
    zc_lat float NOT NULL ,
    zc_lon float NOT NULL
);
\end{lstlisting}
\section*{Datenimport}
\subsection*{Beispiel: Städte-Relation/zip\_coordinates}
\begin{lstlisting}[language=sql]
INSERT INTO zip_coordinates (zc_loc_id,zc_zip, zc_location_name, zc_lat, zc_lon) 
	SELECT gl.loc_id, plz.text_val1, name.text_val1, coord.lat, coord.lon
	FROM geodb_textdata plz
	LEFT JOIN geodb_textdata name ON plz.loc_id = name.loc_id 
	LEFT JOIN geodb_locations gl ON plz.loc_id = gl.loc_id
	LEFT JOIN geodb_coordinates coord ON plz.loc_id = coord.loc_id
	WHERE plz.text_type = 500300000
	AND name.text_type = 500100000 
	AND gl.loc_type IN (100600000,100700000);
\end{lstlisting}
\section*{Erweiterungen}
\subsection*{Beispiel: zusätzliches Feld Vorwahl}
\begin{lstlisting}[language=sql]
SELECT gl.loc_id, plz.text_val1, name.text_val1, vorwahl.text_val1, 
	coord.lat, coord.lon
	FROM geodb_textdata plz
	LEFT JOIN geodb_textdata name ON plz.loc_id = name.loc_id 
	LEFT JOIN geodb_textdata vorwahl ON plz.loc_id = vorwahl.loc_id
	LEFT JOIN geodb_locations gl ON plz.loc_id = gl.loc_id
	LEFT JOIN geodb_coordinates coord ON plz.loc_id = coord.loc_id
	WHERE plz.text_type = 500300000
	AND name.text_type = 500100000 
	AND vorwahl.text_type = 500400000
	AND gl.loc_type IN (100600000,100700000);
\end{lstlisting}
\end{document}